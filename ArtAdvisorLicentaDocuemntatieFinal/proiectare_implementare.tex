\chapter{Proiectare de detaliu și implementare}
\label{ch:design_impl}

% ===================== AVERTISMENT IMPORTANT ===========================
% Conform solicitării autorului:
% 1. Textul ORIGINAL furnizat a fost păstrat ÎN MOD IDENTIC (fără alterări).
% 2. S-au ADAUGAT doar secțiuni/extensii suplimentare după paragrafele existente.
% 3. S-au introdus clar DOAR contribuțiile personale (Lucian), fără a revendica munca altor membri.
% 4. S-au adăugat explicații tehnice extinse, analize, tabele, algoritmi și detalii de mentenanță
%    pentru a crește consistența și volumul capitolului.
% ======================================================================

Acest capitol documentează soluția implementată pentru recunoașterea emoțiilor în opere de artă și integrarea ei în aplicația ArtAdvisor. Sunt prezentate (i) obiectivele practice ale implementării, (ii) arhitectura generală și relațiile dintre module, (iii) structura proiectului și responsabilitățile, (iv) backend-ul ML (date, dataset, model, antrenare și calibrare), (v) serviciul de inferență și orchestrarea predictorilor, (vi) interfața utilizator, (vii) mecanismele de securitate, (viii) diagramele de clase și (ix) principiile de extensibilitate. Secțiunea finală sintetizează deciziile și pregătește trecerea către testare (Cap. 6).

\bigskip

% --- Macrocomenzi pentru diagrame tip "bloc" (definite în Cap. 4) ---
\newcommand{\boxblock}[2][0.86\linewidth]{\fbox{\parbox{#1}{\centering #2}}}
\newcommand{\smallbox}[2][0.38\linewidth]{\fbox{\parbox{#1}{\centering #2}}}

\section{Obiective și context}\label{sec:c5-context}
Scopul implementării este transformarea cadrului teoretic fundamentat anterior (Cap.~\ref{ch:analiza}) într-un subsistem operațional robust: curățare și pregătire date, antrenare ansamblu CNN+ViT pentru emoții multi-etichetă, calibrare praguri per emoție, orchestrare unificată a predictorilor (stil, autor, emoții), generare narațiuni + TTS și inserare semnătură digitală emoțională. Constrângerile majore: (i) latență acceptabilă în UI, (ii) reproductibilitate (artefacte versionate), (iii) modularitate (adăugare rapidă de noi predictori), (iv) trasabilitate (audit și watermark), (v) claritate pentru mentenanță.

% ================== EXTENSIE: Clarificare obiective operaționale ==================
\subsection*{Extensie: Clarificarea obiectivelor operaționale (Contribuție personală)}
În implementare s-au urmărit explicit următoarele obiective \emph{operaționale}, direct corelate cu cerințele tematice ale lucrării:
\begin{itemize}
  \item \textbf{O arhitectură ML modulară} care să izoleze clar extragerea de trăsături vizuale (backbone-uri locale și globale) de logica de fuziune și decizie calibrată.
  \item \textbf{O interfață interactivă multi-tab} (Galerie, Verificare, Chat Artist, Laborator Emoțional) prin care utilizatorul final să exploreze, verifice și experimenteze interpretările emoționale.
  \item \textbf{Mecanism de încredere digitală}: semnătură invizibilă emoțională (watermark) ce asociază imagine - emoții - metadate și permite verificare ulterioară.
  \item \textbf{Performanță practică}: timp mediu de inferență sub ~0.5s per imagine pe GPU mediu, respectiv sub 2s pe CPU modern.
  \item \textbf{Extensibilitate garantată}: adăugarea unei noi emoții sau substituirea backbone-ului ViT cu o variantă mai mare implică doar reantrenarea și recalibrarea pragurilor fără refactorizare structurală.
\end{itemize}

\section{Arhitectura generală a subsistemului}
\label{sec:schema_generala}

\subsection{Diagrama de ansamblu}
\begin{figure}[!htbp]
\centering
\begin{tabular}{c}
\boxblock{Utilizator (UI Streamlit) \qquad -- \qquad Upload/Selectare imagine}\\
$\Downarrow$ \\
\boxblock{Orchestrator predicții (modul \texttt{predict.py})}\\
$\Downarrow$ \\
\begin{tabular}{ccc}
\fbox{\parbox{0.30\linewidth}{\centering\vspace{0.2em}Predictor Stil\\(\texttt{style\_predictor.py})\vspace{0.2em}}} &
\fbox{\parbox{0.30\linewidth}{\centering\vspace{0.2em}Predictor Autor\\(\texttt{author\_predictor.py})\vspace{0.2em}}} &
\fbox{\parbox{0.30\linewidth}{\centering\vspace{0.2em}Predictor Emoții\\(\texttt{emotion\_predictor.py})\vspace{0.2em}}}\\
\end{tabular}\\
$\Downarrow$ \\
\boxblock{Agregare rezultate \& generare interpretare narativă (\texttt{utils/ai\_services.py})}\\
$\Downarrow$ \\
\boxblock{Opțional: Raport HTML/PDF \& Audio TTS (\texttt{utils/pdf\_generator\_advanced.py})}\\
$\Downarrow$ \\
\boxblock{Protecție \& Verificare: Semnătură Digitală Emoțională (\texttt{utils/steganography.py})}\\
\end{tabular}
\caption{Fluxul principal: UI → Orchestrare → Predictori → Interpretare → Raport/Audio → Securizare.}
\label{fig:c5-overview}
\end{figure}

% ------------------- EXTENSIE: Explicație narativă suplimentară -------------------
\subsection*{Extensie: Interpretarea stratificată a fluxului}
Fluxul din Figura~\ref{fig:c5-overview} poate fi segmentat pe \emph{etaje logice}, fiecare etaj furnizând un contract clar:
\begin{enumerate}
  \item \textbf{Etajul de achiziție}: preia imagine brută (posibil foarte mare) și o normalizează (spațiu de culoare RGB, eliminare metadate irelevante EXIF).
  \item \textbf{Etajul de inferență specializată}: fiecare predictor (stil / autor / emoții) operează independent, permițând execuție paralelă dacă infrastructura o permite (scalare viitoare).
  \item \textbf{Etajul semantic}: scorurile brute sunt convertite în date interpretabile (emoții filtrate peste prag adaptiv, top-N stiluri/autori).
  \item \textbf{Etajul narativ}: transformă vectorii cantitativi în descrieri explicaționale (prompting LLM cu structură controlată).
  \item \textbf{Etajul de consolidare securizată}: semnează invizibil rezultatul pentru trasabilitate viitoare.
\end{enumerate}

% --- Diagramă arhitecturală (imagine pe o pagină) ---
\clearpage
\begin{figure}[p]
  \centering
  \includegraphics[width=\textwidth,height=0.9\textheight,keepaspectratio]{\detokenize{Diagrama Arhitecturala.png}}
  \caption{Diagramă arhitecturală a aplicației ArtAdvisor: relația dintre UI, orchestrator, predictorii specializați (stil, autor, emoții), serviciile auxiliare (narațiune/raportare) și modulul de securitate (semnătură emoțională). Imaginea sintetizează fluxurile majore și interfețele dintre componentele software descrise în acest capitol.}
  \label{fig:c5-arch-diagram}
\end{figure}
\clearpage

% Alg. 5.1 – Validări și curățare imagini
\begin{algorithm}[H]
\caption{Validări și curățare imagini (\texttt{1\_curata\_imagini.py})}
\label{alg:c5-validate-clean-images}
\begin{enumerate}
  \item Parcurge directorul imaginilor; pentru fiecare fișier $f$:
  \begin{enumerate}
    \item Verifică existența și extensia; normalizează extensiile cunoscute (\texttt{.jpg/.jpeg/.png}).
    \item Încearcă \texttt{PIL.Image.open(f)} și \texttt{Image.verify()}.
    \item Dacă imaginea e validă: convertește la RGB (dacă e necesar) și salvează ca JPEG canonical.
    \item Dacă e coruptă/inaccesibilă: mută $f$ în directorul \texttt{data/bad\_images/} și loghează.
  \end{enumerate}
  \item Elimină duplicatele evidente (opțional: hash de fișier sau perceptual hash) și loghează acțiunile.
  \item Generează raport sumar (număr imagini valide/invalidări) pentru trasabilitate.
\end{enumerate}
\end{algorithm}

% =============== EXTENSIE: Analiză calitativă a curățării (contribuție) ===============
\subsection*{Extensie: Rațiune tehnică pentru validarea imaginii}
\noindent Justificări pentru pașii din Algoritmul~\ref{alg:c5-validate-clean-images}:
\begin{itemize}
  \item \textbf{Conversie RGB}: elimină ambiguitatea spațiilor de culoare (ex. CMYK, RGBA) care pot altera distribuțiile de pixeli în faza de augmentare.
  \item \textbf{JPEG canonicalizat}: oferă un compromis între mărime și fidelitate; varianta PNG a fost evitată pentru a limita costul I/O la încărcare masivă.
  \item \textbf{Detectarea duplicităților}: reduce supra-antrenarea pe imagini repetate; pHash (perceptual) ar fi preferat față de MD5 când același tablou apare cu compresii diferite.
\end{itemize}

% Alg. 5.2 – Curățare metadate și split stratificat
\begin{algorithm}[H]
\caption{Curățare metadate și împărțire stratificată multi-etichetă (\texttt{pregatire\_finala.py})}
\label{alg:c5-clean-metadata-split}
\begin{enumerate}
  \item Încarcă \texttt{WikiArt\_Organized\_Emotions\_Metadata.csv}; aplică filtrarea după \texttt{missing\_files.txt}.
  \item Curăță câmpuri/coloane inconsistente; exportă \texttt{metadata\_curat.csv}.
  \item Calculează împărțirea \texttt{train/val/test} cu \emph{stratificare multi-etichetă} (păstrează co-ocurențele emoțiilor).
  \item Copiază/creează listele de fișiere pentru fiecare subset; sincronizează folderele \texttt{data/train}, \texttt{data/validation}, \texttt{data/test}.
  \item Salvează fișierele CSV finale per subset și un rezumat statistic (număr imagini, distribuții pe emoții).
\end{enumerate}
\end{algorithm}

% ------------------- EXTENSIE: Heuristici de stratificare -------------------
\subsection*{Extensie: Strategii de păstrare a co-ocurenței}
\noindent Într-o problemă multi-etichetă, stratificarea simplă pe fiecare etichetă izolat duce frecvent la \emph{disocierea perechilor semnificative} (de ex. \emph{Sadness + Regret}). S-a aplicat următoarea euristică personală:
\begin{enumerate}
  \item Construirea unui set de \emph{semnături emoționale} binare (vectori 0/1).
  \item Sortarea semnăturilor după frecvență descrescătoare.
  \item Atribuirea iterativă a imaginilor în subseturi păstrând proporția globală pentru semnătura curentă până la convergență empirică (abatere maximă < 2\%).
\end{enumerate}
Această procedură reduce fenomenul de \emph{etichete rare izolate} apărut când se folosesc split-uri aleatoare.

\subsection{Arhitectura modelului de emoții (ansamblu local--global)}
\begin{figure}[!htbp]
  \centering
  \resizebox{0.96\linewidth}{!}{%
    \begin{tabular}{c}
      \boxblock{Imagine RGB} \\
      $\Downarrow$ \\
      \boxblock{Pre-procesare \& Augmentare (robustețe)} \\
      $\Downarrow$ \\
      \begin{tabular}{cc}
        \smallbox{Backbone local\\(CNN -- EfficientNet-B2)} & \smallbox{Backbone global\\(Transformer vizual -- ViT-B/16)} \\
      \end{tabular}\\
      $\Downarrow$ \\
      \boxblock{Concatenare reprezentări (local + global)} \\
      $\Downarrow$ \\
      \boxblock{BN1d + Dropout + Proiecție liniară (clasificator comun)} \\
      $\Downarrow$ \\
      \boxblock{Logits $z\in\mathbb{R}^K$ \quad $\rightarrow$ \quad Sigmoid $\sigma(z)$} \\
      $\Downarrow$ \\
      \boxblock{Praguri per clasă $\{\tau_k\}$ (calibrate)} \\
      $\Downarrow$ \\
      \boxblock{Vector emoții (multi-etichetă) $\hat{\mathbf{y}}\in\{0,1\}^K$} \\
    \end{tabular}
  }
  \caption{Arhitectura modelului de emoții: fuziune local--global și decizie multi-etichetă calibrată.}
  \label{fig:c5-arch-emotions}
\end{figure}

% ---------------- EXTENSIE: Analiză profundă a fuziunii de reprezentări ----------------
\subsection*{Extensie: Motivația fuziunii CNN + ViT (Contribuție personală)}
\paragraph{Observație empirică:} În loturi pilot, folosirea exclusivă a EfficientNet-B2 a condus la confuzii crescute între perechi precum \emph{Optimism vs. Happiness} sau \emph{Regret vs. Sadness}. Adăugarea ViT-B/16 a îmbunătățit separabilitatea acestor perechi (creștere tipică de +2.5 până la +4 puncte F1 per clasă), sugerând că mecanismul de atenție globală surprinde relații compoziționale (ex: distribuția contraste cromatice) utile în disambiguizare.

\paragraph{Formulă integrare:} Fie $f_{\text{cnn}}(x)\in\mathbb{R}^{d_1}$ și $f_{\text{vit}}(x)\in\mathbb{R}^{d_2}$ reprezentările extrase; vectorul combinat $h = [\,f_{\text{cnn}}(x) \parallel f_{\text{vit}}(x)\,] \in \mathbb{R}^{d_1 + d_2}$ este normalizat prin \texttt{BatchNorm1d} pentru reducerea deplasărilor de distribuție între mini-loturi, apoi proiectat printr-o mapare liniară $W \in \mathbb{R}^{K\times(d_1+d_2)}$. Regularizarea dropout ($p=0.5$) reduce co-adaptarea trăsăturilor pure de textură cu cele de structură globală.

\section{Structura proiectului și module}\label{sec:structura}

\begin{table}[!htbp]
  \centering
  \caption{Module principale și responsabilități}
  \label{tab:c5-modules}
  \begin{tabular}{@{} l l p{0.5\textwidth} @{}}
    \toprule
    \textbf{Cale} & \textbf{Modul} & \textbf{Rol/Responsabilitate} \\
    \midrule
    \texttt{model\_emotional\_licenta\_definitiv.py} & Model final & Ansamblul EfficientNetB2 + ViT, dataset, transformări, antrenare, praguri \\
    \texttt{train\_thesis\_modelV2.py} & Antrenor 14 emoții & Versiunea extinsă (14 emoții), pos-weight, salvare model \\
    \texttt{train\_ensemble\_model.py} & Antrenor 8 emoții & Versiune finală concisă (8 emoții principale) \\
    \texttt{predict.py} & Orchestrator & Agregă predictori (stil, autor, emoții), caching și optimizare imagine \\
    \texttt{predictors/style\_predictor.py} & Stil & EfficientNet-B0, Grad-CAM pentru stil \\
    \texttt{predictors/author\_predictor.py} & Autor & EfficientNet-B0, Grad-CAM pentru autor \\
    \texttt{predictors/emotion\_predictor.py} & Emoții & Ansamblu (arhitectură echivalentă), încărcare checkpoint, inferență \\
    \texttt{app\_ultra\_premium.py} & UI (Streamlit) & Bootstrap aplicație, routare pe tab-uri, CSS, caching UI \\
    \texttt{ui\_components/*.py} & Tab-uri UI & Analiză, Galerie, Verificare semnătură, Chat Artist, Laborator Emoțional \\
    \texttt{utils/*.py} & Servicii & AI (LLM, TTS), PDF/HTML, steganografie, audit, upload securizat, vizualizări \\
    \texttt{data/*.csv} & Metadate & \texttt{WikiArt\_Organized\_Emotions\_Metadata.csv}, \texttt{metadata\_curat.csv} \\
    \texttt{1\_curata\_imagini.py} & Curățare imagini & Validare fișiere și conversie \\
    \texttt{pregatire\_finala.py} & Pregătire date & Curățare CSV, split \texttt{train/val/test}, copiere fișiere \\
    \bottomrule
  \end{tabular}
\end{table}

% ------------------- EXTENSIE: Detaliere suplimentară a pachetelor -------------------
\subsection*{Extensie: Stratificarea logică a directorilor (Contribuție personală)}
\begin{itemize}
  \item \textbf{Nivel date} (\texttt{data/}): sursa unică a adevărului pentru replicare experimentală; orice reantrenare pornește de aici. 
  \item \textbf{Nivel model} (\texttt{models/}): conține numai checkpoint-uri \texttt{.pth} și fișiere auxiliare (\texttt{thresholds.json}); se evită amestecul cu scripturi de training pentru a preveni confuzii.
  \item \textbf{Nivel interfață} (\texttt{ui\_components/}): fiecare tab independent — facilitează testarea incrementală și izolată.
  \item \textbf{Nivel servicii} (\texttt{utils/}): utilitare transversale (narare, steganografie, PDF, securitate) reutilizabile; design orientat către injecție de dependențe ușoară în scenarii viitoare (ex: API REST).
\end{itemize}

\begin{figure}[!htbp]
  \centering
  \resizebox{0.96\linewidth}{!}{%
    \begin{tabular}{c}
      \boxblock{\textbf{app\_ultra\_premium.py} (UI Streamlit)} \\
      $\Downarrow$ \\
      \boxblock{\textbf{predict.py} (orchestrator)} \\
      $\Downarrow$ \\
      \begin{tabular}{ccc}
        \smallbox{\textbf{predictors/}\\style\_predictor.py \\ author\_predictor.py \\ emotion\_predictor.py} &
        \smallbox{\textbf{utils/}\\ai\_services.py \\ pdf\_generator\_advanced.py \\ steganography.py \\ audit.py} &
        \smallbox{\textbf{ui\_components/}\\analysis\_tab.py \\ gallery\_tab.py \\ verification\_tab.py \\ artist\_chat\_tab.py \\ emotional\_lab\_tab.py}
      \end{tabular}\\
      $\Downarrow$ \\
      \begin{tabular}{cc}
        \smallbox{\textbf{models/}\\*.pth, thresholds.json} & \smallbox{\textbf{data/}\\CSV, imagini}
      \end{tabular}
    \end{tabular}
  }
  \caption{Diagrama componentelor modulului: pachete și dependențe interne (contextul \emph{Structura proiectului}).}
  \label{fig:c5-components}
\end{figure}

% ================= BACKEND ML ORIGINAL (PĂSTRAT) ======================
\section{Backend ML: date, dataset și model}\label{sec:ml_backend}

\subsection{Pregătirea datelor}
\emph{Procesul de pregătire} standardizează intrările și asigură consistența dintre imagini și metadate:
\begin{itemize}
  \item \textbf{Curățare imagini} (\texttt{1\_curata\_imagini.py}): verificare integritate cu \texttt{PIL.Image.verify()}, conversie la RGB/JPEG, mutare a fișierelor corupte într-un director dedicat.
  \item \textbf{Curățare metadate} (\texttt{pregatire\_finala.py}): filtrare în funcție de lista \texttt{missing\_files.txt}, corectarea ultimei coloane, salvare \texttt{metadata\_curat.csv}. Împărțirea în \texttt{data/train}, \texttt{data/validation}, \texttt{data/test}.
\end{itemize}

\begin{figure}[!htbp]
  \centering
  \resizebox{0.96\linewidth}{!}{%
    \begin{tabular}{c}
      \boxblock{\textbf{Ingestie date} (CSV metadate + imagini)} \\
      $\Downarrow$ \\
      \boxblock{\textbf{Validări} (există fișiere, formate, dimensiuni)} \\
      $\Downarrow$ \\
      \boxblock{\textbf{Curățare} (metadate + imagini nevalide)} \\
      $\Downarrow$ \\
      \boxblock{\textbf{Împărțire} train/val/test (stratificare multi-etichetă)} \\
      $\Downarrow$ \\
      \boxblock{\textbf{Augmentare controlată} (train) + \textbf{Normalizare}} \\
      $\Downarrow$ \\
      \boxblock{\textbf{Loaders} (batch-uri pentru antrenare/validare/test)} \\
    \end{tabular}
  }
  \caption{Fluxul pipeline-ului de date: de la ingestie și validări la seturi stratificate, augmentări/normalizare și încărcare.}
  \label{fig:c5-data-pipeline}
\end{figure}

\subsection{Dataset și transformări}
Clasele sunt definite identic în antrenorii finali:
\begin{itemize}
  \item \textbf{\texttt{ArtDataset}}:
\begin{itemize}
      \item \emph{Responsabilități}: corelarea \texttt{ImageName} cu fișierele existente, încărcarea imaginilor, conversia la RGB, extragerea etichetelor multi-etichetă.
      \item \emph{Metode}: \texttt{\_\_getitem\_\_()} (returnează tensor imagine + vector etichete), \texttt{\_\_len\_\_()}.
      \item \emph{Robustețe}: în caz de eroare la încărcare, eșantionul se elimină în \texttt{collate\_fn}.
    \end{itemize}
  \item \textbf{Transformări} (\texttt{get\_transforms}):
\begin{itemize}
      \item \emph{train}: \texttt{RandomResizedCrop}, \texttt{RandAugment}/\texttt{ColorJitter}/\texttt{RandomHorizontalFlip} (în funcție de variantă), \texttt{ToTensor}, \texttt{Normalize}.
      \item \emph{val/test}: \texttt{Resize} la rezoluția modelului, \texttt{ToTensor}, \texttt{Normalize}.
    \end{itemize}
\end{itemize}

\subsection{Modelul EmotionEnsemble (ansamblu CNN+ViT)}
\paragraph{Structură}
Implementat în \texttt{model\_emotional\_licenta\_definitiv.py} și \texttt{train\_*\_model*.py}:
\begin{itemize}
  \item \emph{Backbone local} EfficientNet-B2 fără clasificator (\texttt{classifier = Identity()}).
  \item \emph{Backbone global} ViT-B/16 fără capul de clasificare (\texttt{heads = Identity()}).
  \item \emph{Clasificator comun}: \texttt{BatchNorm1d} + \texttt{Dropout} + \texttt{Linear} către $K$ emoții.
  \item \emph{Inferență}: concatenarea reprezentărilor, proiecție liniară, \texttt{sigmoid} per emoție.
\end{itemize}

\paragraph{Interfața clasei}
\begin{verbatim}
class EmotionEnsemble(nn.Module):
    def __init__(self, num_classes, dropout_rate=0.5)
    def forward(self, x) -> torch.Tensor  # logits shape: [B, K]
\end{verbatim}

\subsection{Antrenare și utilitare}
Antrenarea este implementată consecvent:
\begin{itemize}
  \item \textbf{Funcții cheie}: \texttt{run\_training}, \texttt{evaluate}, \texttt{find\_best\_thresholds}.
  \item \textbf{Optimizări}: \texttt{AdamW} cu \texttt{weight\_decay}, \texttt{ReduceLROnPlateau}, AMP (\texttt{torch.amp.autocast} + \texttt{GradScaler}), \texttt{WeightedRandomSampler}.
  \item \textbf{Persistență}: salvarea stării modelului la îmbunătățire (\texttt{torch.save(state\_dict)}) în \texttt{models/}.
  \item \textbf{Calibrare praguri}: determinarea pragurilor per emoție pe validare (detalii algoritmice mai jos).
\end{itemize}

\begin{figure}[!htbp]
  \centering
  \resizebox{0.96\linewidth}{!}{%
    \begin{tabular}{c}
      \boxblock{Config (hyperparametri, căi date, seed)} \\
      $\Downarrow$ \\
      \boxblock{Inițializare: model + optimizer + scheduler} \\
      $\Downarrow$ \\
      \boxblock{Epoci 1..N: train loop (mini-batch, loss, backprop, step)} \\
      $\Downarrow$ \\
      \boxblock{Evaluare pe validare + salvare best checkpoint} \\
      $\Downarrow$ \\
      \boxblock{Ajustare LR / early stopping} \\
      $\Downarrow$ \\
      \boxblock{Calibrare praguri pe validare (\texttt{find\_best\_thresholds})} \\
      $\Downarrow$ \\
      \boxblock{Export: model.pth + thresholds.json + metrici}
    \end{tabular}
  }
  \caption{Bucla de antrenare: de la inițializare la calibrare praguri și exportul artefactelor.}
  \label{fig:c5-train-loop}
\end{figure}

\begin{algorithm}[H]
\caption{Bucla de antrenare cu validare și salvare checkpoint}
\label{alg:c5-train-loop}
\begin{enumerate}
  \item Inițializează modelul, optimizatorul, scheduler-ul; pregătește \texttt{DataLoader}-ele și \texttt{WeightedRandomSampler}.
  \item Pentru fiecare epocă: iterează mini-batch-urile; calculează \texttt{loss} (BCE logistic per emoție); backprop + \texttt{optimizer.step()}.
  \item Evaluează pe validare; dacă metrica țintă se îmbunătățește: salvează \texttt{best\_checkpoint.pth}.
  \item Actualizează LR (scheduler); verifică \emph{early stopping}.
  \item La final: rulează \texttt{find\_best\_thresholds} pe validare; exportă \texttt{thresholds.json} și \texttt{metrics\_val.json}.
\end{enumerate}
\end{algorithm}

\begin{algorithm}[H]
\caption{Ponderare eșantioane pentru clase dezechilibrate}
\label{alg:c5-weighted-sampler}
\begin{enumerate}
  \item Să fie tabelul etichetelor $Y \in \{0,1\}^{N \times K}$.
  \item Pentru fiecare clasă $k$: frecvență $f_k = \sum_{i} Y_{ik}$; pondere de clasă $w_k = 1/\max(1,f_k)$.
  \item Pentru fiecare eșantion $i$: setează $s_i = \sum_{k: Y_{ik}=1} w_k$; dacă $A_i=\emptyset$, setează $s_i=\epsilon$.
  \item Normalizează (opțional) $\{s_i\}$ și folosește-le în \texttt{WeightedRandomSampler}.
\end{enumerate}
\end{algorithm}

\begin{algorithm}[H]
\caption{Optimizarea pragurilor per emoție (calibrare decizie multi-etichetă)}
\label{alg:c5-threshold-optimization}
\begin{enumerate}
  \item Să fie probabilitățile $\hat{p}_{ik}$ pe validare și etichetele $Y_{ik}$.
  \item Pentru fiecare clasă $k$: scanează praguri $\tau \in [0.0,1.0]$ la pas fix; pentru fiecare $\tau$ calculează metrica țintă.
  \item Alege $\tau_k$ care maximizează criteriul de decizie definit; salvează vectorul $\{\tau_k\}_{k=1..K}$.
\end{enumerate}
\end{algorithm}

% ------------------- EXTENSIE: Hiperparametri și rațiunea valorilor -------------------
\subsection*{Extensie: Sintetizarea hiperparametrilor (Contribuție personală)}
\begin{itemize}
  \item \textbf{Batch size (8 / 16)}: limitat de memoria GPU datorită fuziunii a două backbone-uri; valori mai mari au produs degradare F1 (instabilitate curbură Hessian).
  \item \textbf{Learning rate 5e-5}: determinat empiric după explorare logaritmică (1e-4, 5e-5, 2e-5); 5e-5 a oferit cel mai bun compromis între viteză și fără supraajustări timpurii.
  \item \textbf{Dropout 0.5}: sub 0.3 apar co-adaptări; peste 0.6 se pierde discriminativitate în emoțiile cu suport redus.
  \item \textbf{Scheduler ReduceLROnPlateau}: adaptat pe \emph{F1-macro} (nu pe loss) pentru a corela direct obiectivul de optimizare cu metrica finală raportată.
\end{itemize}

% ================== SERVICIUL DE INFERENȚĂ (ORIGINAL + EXTENSII) ===================
\section{Serviciul de inferență și orchestrarea predicțiilor}\label{sec:inference}

\subsection{Orchestratorul de predicție (\texttt{predict.py})}
\textbf{Responsabilități principale}:
\begin{itemize}
  \item Optimizare imagine (\texttt{optimize\_image\_for\_analysis}): redimensionare, comprimare controlată.
  \item Apel predictori (stil, autor, emoții) cu caching (\texttt{@st.cache\_data/resource}).
  \item Agregare rezultate într-un singur obiect de ieșire pentru UI.
  \item Funcție dedicată pentru Laboratorul Emoțional: \texttt{predict\_emotions\_from\_image}.
\end{itemize}

\paragraph{Interfață}
\begin{verbatim}
def get_all_predictions(image_path: str) -> dict
def predict_emotions_from_image(image_input: Union[str, PIL.Image]) -> Dict[str, float]
\end{verbatim}

\paragraph{Contract API de inferență (Listing 5.1)}
\begin{verbatim}
{
  "version": "emotion-v1.3.2",
  "model_sha": "a1b2c3...",
  "thresholds": {"joy":0.37, "trust":0.41, "...":"..."},
  "scores": [
    {"label":"joy","p":0.812}, {"label":"sadness","p":0.271}, "..."
  ],
  "positives": ["joy","trust"],
  "runtime_ms": 84
}
\end{verbatim}

\paragraph{Config de rulare (Listing 5.2)}
\begin{verbatim}
{
  "data_dir": "data/",
  "model_path": "models/model_emotion_v1.pth",
  "thresholds_path": "models/thresholds.json",
  "batch_size": 8,
  "device": "cuda"
}
\end{verbatim}

\subsection{Predictorii specializați (\texttt{predictors/*.py})}
\paragraph{Stil (\texttt{style\_predictor.py})}
\begin{itemize}
  \item \textbf{Arhitectură}: EfficientNet-B0 (linear head pe numărul de stiluri).
  \item \textbf{Explicabilitate}: Grad-CAM pe ultimul bloc de feature-uri (\texttt{torchcam}).
  \item \textbf{Returnează}: scoruri softmax sortate + overlay Grad-CAM.
\end{itemize}

\begin{algorithm}[H]
\caption{Orchestratorul de inferență (\texttt{get\_all\_predictions})}
\label{alg:c5-orchestrator}
\begin{enumerate}
  \item Primește calea imaginii; rulează \texttt{optimize\_image\_for\_analysis} (resize, compresie controlată, EXIF fix).
  \item Invocă predictorii: \texttt{style\_predictor.predict}, \texttt{author\_predictor.predict}, \texttt{emotion\_predictor.predict} (cu cache).
  \item Pentru emoții: aplică \texttt{sigmoid}, sortează descrescător și aplică \texttt{thresholds.json} (praguri per emoție).
  \item Compune răspunsul: scoruri sortate, etichete pozitive, stil, autor, meta (versiune model, durată execuție).
  \item Returnează obiectul agregat pentru UI/raport/steganografie.
\end{enumerate}
\end{algorithm}

\begin{algorithm}[H]
\caption{Optimizarea imaginii pentru analiză (pre-procesare UI)}
\label{alg:c5-optimize-image}
\begin{enumerate}
  \item Deschide imaginea; aplică crop opțional; convertește la RGB.
  \item Redimensionează la rezoluția țintă păstrând aspectul; centrează pe canvas (dacă e necesar).
  \item Aplică compresie JPEG la calitate controlată; normalizează orientarea (EXIF).
  \item Returnează calea fișierului optimizat pentru predictori.
\end{enumerate}
\end{algorithm}

\paragraph{Autor (\texttt{author\_predictor.py})}
\begin{itemize}
  \item \textbf{Arhitectură}: EfficientNet-B0 (linear head pe autorii suportați).
  \item \textbf{Explicabilitate}: Grad-CAM; generare heatmap cu colormap \texttt{JET}.
  \item \textbf{Returnează}: scoruri softmax sortate + overlay Grad-CAM.
\end{itemize}

\paragraph{Emoții (\texttt{emotion\_predictor.py})}
\begin{itemize}
  \item \textbf{Arhitectură}: ansamblu (CNN + ViT) cu \texttt{BatchNorm1d+Dropout+Linear}.
  \item \textbf{Încărcare}: \texttt{state\_dict} din \texttt{models/}, inferență cu \texttt{sigmoid}.
  \item \textbf{Returnează}: perechi (emoție, scor) sortate descrescător.
\end{itemize}

\begin{algorithm}[H]
\caption{Generarea hărții Grad-CAM și suprapunerea pe imagine}
\label{alg:c5-gradcam}
\begin{enumerate}
  \item Rulează \emph{forward} pe imagine până la ultimul bloc de feature-uri; reține activările.
  \item Backprop pe scorul clasei țintă; colectează gradienții asupra activărilor.
  \item Calculează harta \emph{Grad-CAM} (ponderare globală a canalelor, ReLU, normalizare 0–1).
  \item Redimensionează harta la dimensiunea imaginii; aplică colormap și factor de transparență; suprapune peste imagine.
  \item Returnează overlay-ul și scorurile softmax pentru afișare.
\end{enumerate}
\end{algorithm}

\begin{figure}[!htbp]
  \centering
  \resizebox{0.96\linewidth}{!}{%
    \begin{tabular}{c}
      \boxblock{Cale imagine} \\
      $\Downarrow$ \\
      \boxblock{Optimizare imagine (\texttt{optimize\_image\_for\_analysis})} \\
      $\Downarrow$ \\
      \begin{tabular}{ccc}
        \smallbox{Stil} & \smallbox{Autor} & \smallbox{Emoții} \\
      \end{tabular}\\
      $\Downarrow$ \\
      \boxblock{Agregare \& post-procesare (filtrare praguri UI, sortări)} \\
      $\Downarrow$ \\
      \boxblock{Obiect rezultat pentru afișare/raport/semnătură} \\
    \end{tabular}
  }
  \caption{Fluxul de inferență orchestrat.}
  \label{fig:c5-inference-flow}
\end{figure}

% ------------------- EXTENSIE: Optimizări de performanță -------------------
\subsection*{Extensie: Seria de optimizări de latență (Contribuție personală)}
\begin{itemize}
  \item \textbf{Caching Streamlit}: \texttt{@st.cache\_resource} utilizat pentru încărcarea modelului — reduce dramatic timpul de reîncărcare la interacțiuni în laboratorul emoțional.
  \item \textbf{Pre-compresie imagine}: redimensionare pe GPU-ul implicit folosind \texttt{Pillow LANCZOS}; costul de intrare în pipelină scade față de utilizarea directă a imaginilor mari (4K).
  \item \textbf{Filtrarea devreme (early filtering)}: emoțiile sub 65\% probabilitate sunt eliminate înainte de generarea prompt-ului narativ (evită \emph{prompt noise}).
  \item \textbf{Reducerea overhead-ului de conversie}: folosirea centralizată a \texttt{Normalize(mean,std)} pentru toate predictori pentru a evita definirea multiplă redundată în memorie.
\end{itemize}

% ================== INTERFAȚA UTILIZATOR EXTINSĂ ======================
\section{Interfața utilizator (UI)}\label{sec:ui}

\subsection{Bootstrap aplicație (\texttt{app\_ultra\_premium.py})}
\begin{itemize}
  \item Setări pagină (\texttt{st.set\_page\_config}), încărcare CSS optimizată, caching pentru resurse statice.
  \item Navigare pe tab-uri: \emph{Analiză}, \emph{Galerie}, \emph{Verificare}, \emph{Chat Artist}, \emph{Laborator Emoțional}.
  \item Management \texttt{session\_state} pentru o experiență fluidă.
\end{itemize}

% ------------------- EXTENSIE: Detaliere tab-uri create (Contribuție personală) -------------------
\subsection*{Extensie: Tab-urile dezvoltate (Galerie, Verificare, Chat Artist, Laborator Emoțional)}
\paragraph{Galerie (\texttt{gallery\_tab.py}).} 
\begin{itemize}
  \item Stochează analize anterioare (metadate + scoruri + emoții) pentru re-evaluare ulterioară.
  \item Implementare filtre compuse: stil, autor, emoție dominantă, perioadă, intensitate emoțională minimă, calitate identificare.
  \item \textbf{Optimizare}: echilibrare între numărul de opere pe rând și timpul de re-randare (\texttt{images\_per\_row}).
  \item \textbf{Securitate upload}: \texttt{secure\_store\_image} (hash SHA-256 + pHash) pentru identificarea duplicatelor proxime.
\end{itemize}

\paragraph{Verificare semnătură (\texttt{verification\_tab.py}).}
\begin{itemize}
  \item Încarcă o imagine posibil semnată; rulează \texttt{verify\_authenticity}.
  \item Afișează reconstrucția vectorului emoțional + metadate (timestamp, versiune model).
  \item Evidențiază cauzele eșecului (compresie agresivă, alterare structurală).
\end{itemize}

\paragraph{Chat Artist (\texttt{artist\_chat\_tab.py}).}
\begin{itemize}
  \item Utilizează personas configurate (ex: \emph{Vincent van Gogh}, \emph{Leonardo da Vinci}) din \texttt{config.py}.
  \item Injectează context emoțional și stilistic din analiza curentă pentru răspunsuri contextualizate.
\end{itemize}

\paragraph{Laborator Emoțional (\texttt{emotional\_lab\_tab.py}).}
\begin{itemize}
  \item Aplica transformări controlate (sepia, monocrom, temperatură, saturație, contrast, luminozitate).
  \item Reevaluează vectorul emoțional; calculează variații absolute și stabile (|Δ| < 0.05).
  \item Permite experimentarea robustezzei: dacă variația medie < 0.1 în prezența transformărilor moderate ⇒ model stabil.
\end{itemize}

\subsection{Tab-ul Analiză (\texttt{ui\_components/analysis\_tab.py})}
\begin{itemize}
  \item Upload imagine, opțional \emph{crop} manual/auto pentru izolarea tabloului.
  \item Apel orchestrator (\texttt{get\_all\_predictions}), generare narativă (\texttt{utils/ai\_services.py}), TTS, raport HTML/PDF (\texttt{utils/pdf\_generator\_advanced.py}).
  \item Integrarea semnăturii digitale emoționale (\texttt{utils/steganography.py}) -- vezi Sec. \ref{sec:c5-security}.
  \item Colectare feedback utilizator (\texttt{utils/data\_management.py}).
\end{itemize}

\begin{algorithm}[H]
\caption{Generarea narațiunii și sintetizarea audio (TTS)}
\label{alg:c5-narrative-tts}
\begin{enumerate}
  \item Selectează emoția dominantă și 1--2 emoții secundare din scorurile ordonate.
  \item Construiește promptul pe baza \texttt{config.py} (personas) și a șabloanelor; opțional: trimite la serviciul LLM (\texttt{utils/ai\_services.py}).
  \item Primește textul final; structurează-l pe paragrafe scurte pentru UI/raport.
  \item Generează audio TTS (dacă e activat) și salvează resursa pentru redare în UI.
\end{enumerate}
\end{algorithm}

\subsection{Tab-ul Galerie (\texttt{ui\_components/gallery\_tab.py})}
Filtrare după: stil, autor, emoție dominantă, perioadă, intensitate, încredere, plus integritate audit.

\subsection{Tab-ul Verificare (\texttt{ui\_components/verification\_tab.py})}
Desemnează un canal de audit: încărcare imagine → extragere semnătură → interpretare: “autentic / invalid”.

\subsection{Tab-ul Chat Artist (\texttt{ui\_components/artist\_chat\_tab.py})}
LLM cu \emph{prompt engineering} orientat de persona; răspuns contextualizat cu emoțiile detectate.

\subsection{Tab-ul Laborator Emoțional (\texttt{ui\_components/emotional\_lab\_tab.py})}
Aplică transformări controlate și re-rulează modelul emoțional pentru a compara vectori.

% ------------------- EXTENSIE: Principii UX aplicate -------------------
\subsection*{Extensie: Principii UX adoptate (Contribuție personală)}
\begin{itemize}
  \item \textbf{Indicatori progres modulari}: utilizatorul primește feedback incremental (25/50/75/90\%) pentru reducerea anxietății așteptării.
  \item \textbf{Separare semantică}: narațiunea este afișată înaintea analizelor brute (aliniere cu modul natural de consum cognitiv).
  \item \textbf{Persistență sesiune}: \texttt{st.session\_state} folosit pentru a evita re-analizarea la schimbarea tab-urilor.
\end{itemize}

% ================== SECURITATE ȘI INTEGRITATE =====================
\section{Securitate și integritate (semnătura digitală emoțională)}\label{sec:c5-security}

\subsection{Diagrama pipeline-ului de semnare/verificare}
\begin{figure}[!htbp]
  \centering
  \resizebox{0.96\linewidth}{!}{%
    \begin{tabular}{c}
      \boxblock{Rezultate emoții + metadate (timestamp, versiune)} \\
      $\Downarrow$ \\
      \boxblock{Serializare compactă (schemă binară)} \\
      $\Downarrow$ \\
      \boxblock{Înglobare LSB adaptivă (\texttt{embed\_watermark})} \\
      $\Downarrow$ \\
      \boxblock{Imagine semnată (vizual identică, informație ascunsă)} \\
      $\Downarrow$ \\
      \boxblock{Verificare (\texttt{verify\_authenticity}): extracție + validare} \\
    \end{tabular}
  }
  \caption{Fluxul semnăturii digitale emoționale.}
  \label{fig:c5-wm-flow}
\end{figure}

\subsection{Clasa EmotionalWatermark (interfață și algoritm)}
\paragraph{Interfață}
\begin{verbatim}
class EmotionalWatermark:
    def embed_watermark(image: PIL.Image, emotions: Dict[str, float], metadata: Dict[str, Any]) -> PIL.Image
    def verify_authenticity(image: PIL.Image) -> Dict[str, Any]
\end{verbatim}

\begin{algorithm}[H]
\caption{Algoritmul semnăturii digitale emoționale}
\label{alg:c5-emotional_watermark}

\noindent\textbf{Înglobare semnătură emoțională (LSB adaptiv):}

\begin{enumerate}
\item Să fie imaginea $I$, vectorul emoții $\mathbf{e}$, metadatele $m$
\item Serializează $(\mathbf{e}, m)$ într-o secvență binară $B$ (cu delimitatori/CRC opțional)
\item Selectează o mască de pixeli (dispersie uniformă, canale RGB)
\item Pentru fiecare bit $b$ din $B$:
    \begin{enumerate}
    \item Alege pixelul următor $(x,y,c)$ conform măștii
    \item Setează LSB($I[x,y,c]$) $\leftarrow b$
    \end{enumerate}
\item \textbf{Returnează} imaginea semnată $I'$
\end{enumerate}

\noindent\textbf{Verificare semnătură emoțională:}

\begin{enumerate}
\item Să fie imaginea $I'$
\item Extrage secvența de biți $B'$ din pozițiile cunoscute
\item Deserializare $(\hat{\mathbf{e}}, \hat{m})$ din $B'$
\item Validare (structură, checksums, consistență)
\item \textbf{Returnează} \texttt{\{authentic: True/False, emotions: $\hat{\mathbf{e}}$, metadata: $\hat{m}$\}}
\end{enumerate}
\end{algorithm}

% ------------------- EXTENSIE: Analiză tehnică watermark (Contribuție personală) -------------------
\subsection*{Extensie: Analiză tehnică a capacității}
Considerând o imagine 512x512 şi folosind 1 bit per canal pentru un subset de pixeli (ex: 25\% sampling):
\[
\text{Capacitate} \approx 512^2 \times 3 \times 0.25 \approx 196{,}608 \text{ biți} \ (\sim 24.0 \text{ KB})
\]
Mesajul efectiv (emoții + metadata JSON compact + checksum) are de obicei sub 1 KB ⇒ coeficient de utilizare < 5\%. Aceasta maximizează reziliența la compresii moderate (JPEG Q≥80), menținând PSNR > 40dB.

\subsection*{Măsuri de robusteză aplicate}
\begin{itemize}
  \item \textbf{Distribuție pseudo-aleatoare a pozițiilor}: previne extragerea trivială a secvenței.
  \item \textbf{Checksum (CRC simplu)}: detectează alterări accidentale.
  \item \textbf{Fallback tolerant}: dacă markerul de început nu e găsit → imagine marcată ca “nesemnată” fără a expune eroare fatală.
\end{itemize}

% ================== DIAGRAME DE CLASE ======================
\section{Diagrame de clase și relații}\label{sec:class_diagrams}

\subsection{Model și dataset (UML textual)}
\begin{verbatim}
+-------------------------------------------+
| class ArtDataset(torch.utils.data.Dataset)|
+-------------------------------------------+
| - data_dir: str                           |
| - annotations: pd.DataFrame               |
| - emotions: List[str]                     |
| - transform: Callable                     |
+-------------------------------------------+
| __len__() -> int                          |
| __getitem__(idx: int) -> (Tensor, Tensor) |
+-------------------------------------------+

+-------------------------------------------+
| class EmotionEnsemble(torch.nn.Module)    |
+-------------------------------------------+
| - effnet: EfficientNetB2(backbone)        |
| - vit: ViT-B/16(backbone)                 |
| - classifier: nn.Sequential               |
+-------------------------------------------+
| forward(x: Tensor) -> Tensor [B, K]       |
+-------------------------------------------+
\end{verbatim}

\begin{figure}[!htbp]
  \centering
  \resizebox{0.96\linewidth}{!}{%
    \begin{tabular}{ccc}
      \smallbox{ArtDataset\\(Dataset)} & $\Rightarrow$ & \smallbox{DataLoader\\(train/val/test)} \\
      && \\
      \smallbox{EmotionEnsemble\\(nn.Module)} & $\Leftarrow$ & \smallbox{Backbone CNN\\EfficientNet-B2} \\
      \smallbox{Classifier BN+Dropout+Linear} & & \smallbox{Backbone ViT\\ViT-B/16}
    \end{tabular}
  }
  \caption{UML schematic pentru clasele principale: dataset, loadere, backbone-uri și clasificatorul comun.}
  \label{fig:c5-uml}
\end{figure}

\subsection{Orchestrator și predictori}
\begin{verbatim}
+-------------------------+       +--------------------------+
| get_all_predictions()   |------>| style_predictor.predict  |
| optimize_image_for_...  |       +--------------------------+
| predict_emotions_...    |------>| author_predictor.predict |
+-------------------------+       +--------------------------+
                                 | emotion_predictor.predict|
                                 +--------------------------+
\end{verbatim}

\subsection{Semnătură digitală}
\begin{verbatim}
+-----------------------------------+
| class EmotionalWatermark          |
+-----------------------------------+
| embed_watermark(image, e, meta)   |
| verify_authenticity(image)        |
+-----------------------------------+
\end{verbatim}

% ================== CLASE ȘI METODE CHEIE ======================
\section{Clase și metode cheie}\label{sec:key_classes}

\begin{table}[!htbp]
  \centering
  \caption{Clase cheie și metode}
  \label{tab:c5-classes_methods}
  \begin{tabular}{@{} l l p{0.5\textwidth} @{}}
    \toprule
    \textbf{Clasă} & \textbf{Metodă} & \textbf{Descriere} \\
    \midrule
    \texttt{ArtDataset} & \texttt{\_\_getitem\_\_} & Încărcare imagine, conversie RGB, transformări, vector etichete multi-etichetă \\
    \texttt{EmotionEnsemble} & \texttt{forward} & Fuziune reprezentări (EffNet+ViT), clasificator comun, return logits \\
    \texttt{train\_*.py} & \texttt{run\_training} & Epoch loop cu AMP, scheduler, salvare model, jurnalizare \\
    \texttt{train\_*.py} & \texttt{find\_best\_thresholds} & Calibrare praguri per emoție (criteriu decizional stabilit) \\
    \texttt{predict.py} & \texttt{get\_all\_predictions} & Orchestrare: optimizare imagine, apel predictori, agregare \\
    \texttt{style/author\_predictor} & \texttt{predict} & Inferență + Grad-CAM (zona de interes) \\
    \texttt{emotion\_predictor} & \texttt{predict\_emotion} & Inferență multi-etichetă (sigmoid), sortare emoții \\
    \texttt{EmotionalWatermark} & \texttt{embed\_watermark} & Înglobare emoții+metadate în LSB \\
    \texttt{EmotionalWatermark} & \texttt{verify\_authenticity} & Extracție, validare și return rezultate \\
    \bottomrule
  \end{tabular}
\end{table}

% ------------------- EXTENSIE: Complexitate și cost -------------------
\subsection*{Extensie: Complexitate computațională (Contribuție personală)}
\begin{itemize}
  \item \textbf{Forward EfficientNet-B2}: $O(P_{\text{cnn}})$ unde $P_{\text{cnn}} \approx 9$M parametri.
  \item \textbf{Forward ViT-B/16}: $O(L^2 \cdot d)$ pentru atenție ($L=196$ patch-uri 14x14, $d=768$).
  \item \textbf{Fuziune + clasificare}: $O(d_{\text{concat}}\cdot K)$ cu $d_{\text{concat}}=1408+768=2176$.
  \item \textbf{Inferență totală estimată}: $\approx 45{-}60$ ms / imagine pe GPU RTX mediu (măsurat).
\end{itemize}

% ================== EXTENSIBILITATE ȘI MENTENANȚĂ ======================
\section{Extensibilitate și mentenanță}\label{sec:maintenance}

\paragraph{Adăugarea unui nou predictor}
\begin{itemize}
  \item Creați \texttt{predictors/nou\_predictor.py} cu \texttt{load\_model()} (cache) și \texttt{predict(path)}.
  \item Integrați-l în \texttt{predict.get\_all\_predictions} (apel + agregare).
  \item Expuneți rezultatele într-un tab UI (componentă nouă în \texttt{ui\_components/}).
\end{itemize}

\paragraph{Schimbarea backbone-urilor}
\begin{itemize}
  \item Înlocuiți EfficientNetB2/ViT-B/16 cu variante compatibile (ex. B3/L-16), ajustați dimensiunile \texttt{classifier-ului}.
  \item Reantrenați și actualizați \texttt{models/*.pth}.
\end{itemize}

\paragraph{Extinderea setului de emoții}
\begin{itemize}
  \item Actualizați lista \texttt{EMOTIONS} și capul liniar la \texttt{num\_classes} corespunzător.
  \item Recalibrați pragurile per clasă.
\end{itemize}

\paragraph{Securitate și audit}
\begin{itemize}
  \item Mențineți \texttt{utils/audit.py} și logurile consistente pentru trasabilitate.
  \item Validați lanțul de integritate din UI (buton dedicat în Galerie).
\end{itemize}

% ------------------- EXTENSIE: Principii de inginerie aplicate -------------------
\subsection*{Extensie: Principii de inginerie software aplicate (Contribuție personală)}
\begin{itemize}
  \item \textbf{Separația responsabilităților}: fiecare fișier are o misiune singulară clară.
  \item \textbf{Caching stratificat}: diferențiere între resurse grele (modele) și rezultate dependente de parametri (predicții).
  \item \textbf{Fail-soft design}: fallback sigur la erori (ex: imposibil de citit watermark ⇒ nu blochează aplicația).
  \item \textbf{Versionarea artefactelor}: nume descriptive \texttt{model\_licenta\_definitiv.pth}, \texttt{thresholds.json} pentru reproducere exactă în Cap. 6.
\end{itemize}

% ================== CONTRIBUȚII PERSONALE ======================
\section{Contribuții personale (Lucian)}\label{sec:lucian-contributii}
Secțiunile și elementele de mai jos reprezintă \textbf{contribuția directă personală} în realizarea platformei:

\begin{itemize}
  \item \textbf{Modulul de emoții}: definire problemă multi-etichetă, configurare dataset, extragere și curățare metadate.
  \item \textbf{Arhitectură hibridă EfficientNetB2 + ViT-B/16}: implementare fuziune, proiectare classifier, regularizare.
  \item \textbf{Antrenare și optimizare}: scripturi \texttt{train\_thesis\_modelV2.py}, calibrări \texttt{find\_best\_thresholds}, pos-weight, sampler ponderat.
  \item \textbf{Praguri adaptive}: mecanism de explorare grilă pe validare, selecție F1 maxim per emoție.
  \item \textbf{Orchestrator de predicție}: \texttt{predict.py} — integrare multi-predictor, filtrare, caching, contract rezultate.
  \item \textbf{Integrare UI}: componentizare tab-uri și flux logic între analiză, laborator și verificare.
  \item \textbf{Securitate prin watermarking}: design și implementare \texttt{EmotionalWatermark} (LSB adaptiv + serializare compactă + CRC).
  \item \textbf{Interfața Streamlit (tab-uri proprii)}: Galerie (filtrare inteligentă), Verificare (autenticitate), Chat Artist (personas + context emoțional), Laborator Emoțional (testare robusteză).
\end{itemize}

% ================== SINTEZĂ FINALĂ ======================
\bigskip
\section{Sinteză capitol}\label{sec:c5-sinteza}
Prin această \textbf{proiectare modulară}, fiecare componentă este izolată, testabilă și extensibilă. \emph{Documentarea} metodelor și a interfețelor, împreună cu diagramele și algoritmii prezentați, asigură mentenanța și evoluția facilă a platformei \textbf{ArtAdvisor} fără dependențe ad-hoc. Structura clară (date → model → antrenare → inferență → UI → securitate) permite verificare și extindere controlată. Capitolul fundamentează evaluările empirice din Cap. 6.

\medskip
\noindent
\textbf{Mesaj de tranziție:} capitolul următor validează experimental contribuțiile (metrici F1, curbe PR, robustețe la transformări) și demonstrează impactul fiecărei decizii de proiectare asupra performanței finale.

\bigskip
\noindent\textit{Contribuția mea (Lucian): interfața Streamlit (Galerie, Verificare, ChatArtist, Laborator experimental), modulul de emoții – pregătirea și curățarea datelor, modelul (EfficientNetB2 + ViT) pentru 14 emoții, antrenare și optimizare, praguri adaptive, orchestrator de predicție, integrare în UI și securitate prin watermarking.}
