\chapter{Obiectivele proiectului}
\label{ch:obiective}

\section{Formularea temei}
\label{sec:obiective-tema}

\textbf{Tema propusă:} Dezvoltarea unei platforme computaționale pentru recunoașterea emoțiilor în opere de artă folosind \textbf{inteligența artificială}. 

Platforma va permite:
\begin{itemize}
    \item \textbf{Analiza picturilor} și identificarea emoțiilor pe care le transmit
    \item \textbf{Oferirea unei experiențe interactive și educative} în înțelegerea artei digitale
    \item \textbf{Prezentarea rezultatelor} într-un format accesibil și explicabil
\end{itemize}

Proiectul vizează \textbf{îmbinarea tehnologiei moderne cu aprecierea artistică}, facilitând accesul publicului larg la interpretarea și înțelegerea operelor de artă prin intermediul tehnologiei.

\subsection*{Soluția propusă oferă}
\begin{enumerate}
  \item \textbf{Analiză emoțională completă} - identificarea și evaluarea emoțiilor transmise de fiecare operă de artă
  \item \textbf{Prezentare accesibilă} - rezultate prezentate în mod vizual și narativ pentru înțelegere facilă
  \item \textbf{Instrumente de validare} - mecanisme pentru verificarea și testarea robusteții analizelor efectuate
\end{enumerate}

\section{Obiectiv general}
\label{sec:obiectiv-general}

\textbf{Obiectivul principal} al acestui proiect constă în realizarea unei platforme \textbf{ArtAdvisor} care să permită \textbf{analiza emoțională} a operelor de artă într-o manieră \textbf{accesibilă și precisă}.

Platforma combină \textbf{rigoarea științifică} cu \textbf{claritatea prezentării}, oferind utilizatorilor posibilitatea să înțeleagă și să aprecieze operele de artă din perspectiva emoțională pe care acestea o transmit.

\subsection*{Prin intermediul platformei ArtAdvisor, utilizatorii vor avea acces la}
\begin{itemize}
  \item \textbf{Analiză emoțională precisă și comprehensivă} bazată pe algoritmi de învățare automată
  \item \textbf{Explicații clare și ușor de înțeles} pentru rezultatele obținute
  \item \textbf{Instrumente de validare} pentru verificarea și testarea rezultatelor
  \item \textbf{Interfață interactivă} cu multiple modalități de explorare a datelor
\end{itemize}

\section{Obiective specifice}
\label{sec:obiective-specifice}

Obiectivele specifice ale prezentei lucrări, centrate pe \textbf{contribuția personală a autorului} în domeniul recunoașterii emoțiilor, sunt organizate pe următoarele categorii:

\subsection{Analiza și interpretarea emoțiilor (Contribuție personală)}
\begin{itemize}
  \item \textbf{Identificarea emoțiilor} - recunoașterea unui set de 14 emoții distincte în operele de artă folosind arhitectura local-global dezvoltată
  \item \textbf{Evaluarea intensității} - determinarea gradului de expresivitate pentru fiecare emoție identificată prin scoruri calibrate
  \item \textbf{Profilul emoțional multi-etichetă} - permiterea co-existenței multiple emoții într-o singură operă
\end{itemize}

\subsection{Dezvoltarea modelului de recunoaștere (Contribuție personală)}
\begin{itemize}
  \item \textbf{Arhitectura sistemului} - crearea unui model capabil să analizeze \textbf{caracteristicile locale și globale} ale imaginilor prin fuziunea CNN + Transformer
  \item \textbf{Calibrarea deciziilor} - stabilirea unor \textbf{criterii adaptive} pentru identificarea corectă a emoțiilor dominante prin praguri per clasă
  \item \textbf{Antrenarea robustă} - implementarea strategiilor de antrenare care să gestioneze dezechilibrele din date
\end{itemize}

\subsection{Validarea și testarea sistemului (Contribuție personală)}
\begin{itemize}
  \item \textbf{Protocoale de testare} - implementarea unor metode comprehensive de evaluare a performanței (F1-macro, Average Precision)
  \item \textbf{Testarea robusteții} - verificarea stabilității sistemului în condițiile modificărilor \textbf{foto-realiste}
  \item \textbf{Prezentarea rezultatelor} - dezvoltarea unor modalități \textbf{clare și intuitive} de comunicare a analizelor
\end{itemize}

\subsection{Interfața și experiența utilizatorului (Contribuție personală)}
\begin{itemize}
  \item \textbf{Organizarea funcționalităților} - structurarea platformei în \textbf{module distincte} pentru diferite tipuri de analize:
  \begin{itemize}
    \item \emph{Galerie} - explorarea și organizarea operelor după criterii emoționale
    \item \emph{Verificare} - validarea și testarea analizelor emoționale
    \item \emph{Laborator Emoțional} - experimentare avansată și analiză detaliată
  \end{itemize}
  \item \textbf{Accesibilitatea} - asigurarea unor opțiuni \textbf{diverse de prezentare} a rezultatelor (vizual, audio prin TTS, export)
\end{itemize}

\subsection{Integrarea cu sistemul colaborativ}
\textbf{Nota:} Următoarele funcționalități sunt implementate de colegul David Iakabos și menționate aici pentru contextul complet al sistemului:
\begin{itemize}
  \item Clasificarea stilurilor artistice și identificarea autorilor
  \item Generarea interpretărilor narative prin AI externe 
  \item Chat-ul interactiv cu expertul artificial
\end{itemize}

\subsection{Cerințe funcționale}
\label{subsec:obj-functionale}

\textbf{Platforma ArtAdvisor trebuie să îndeplinească următoarele funcționalități:}

\begin{itemize}
  \item \textbf{Analiza emoțională} - \emph{identificarea și evaluarea} emoțiilor din opere de artă cu posibilitatea detectării \textit{multiple emoții simultane}
  \item \textbf{Generarea profilurilor emoționale} - crearea unei \emph{reprezentări complete} a impactului emoțional al fiecărei opere
  \item \textbf{Explicarea rezultatelor} - oferirea de \textit{justificări vizuale} pentru deciziile sistemului privind emoțiile identificate  
  \item \textbf{Narațiunea automată} - generarea de \emph{descrieri textuale} comprehensive ale analizelor efectuate
  \item \textbf{Verificarea autenticității} - implementarea unui sistem de \textit{semnătură digitală} pentru validarea operelor analizate
  \item \textbf{Organizarea rezultatelor} - dezvoltarea unei \emph{galerii interactive} cu opțiuni de filtrare după criterii multiple
\end{itemize}

\subsection{Cerințe non-funcționale}
\label{subsec:obj-nonfunctionale}

\textbf{Sistemul trebuie să respecte următoarele criterii de calitate:}

\begin{itemize}
  \item \textbf{Performanță} - \emph{timpul de analiză} pentru o imagine să nu depășească 15 secunde după inițializarea sistemului
  \item \textbf{Reproductibilitate} - \textit{rezultatele analizelor} să fie constante și predictibile pentru aceeași operă de artă
  \item \textbf{Robustețe} - sistemul să mențină \emph{stabilitatea performanței} chiar și în condițiile modificărilor minore ale imaginii
  \item \textbf{Securitate și integritate} - implementarea unui sistem de \textit{protecție} care să nu afecteze vizual opera originală
  \item \textbf{Accesibilitate} - interfața să fie \emph{intuitivă și ușor de utilizat}, cu opțiuni diverse de prezentare a rezultatelor
  \item \textbf{Transparență} - sistemul să ofere \textit{informații clare} despre metodele de analiză și gradul de certitudine al rezultatelor
\end{itemize}

\section{Criterii de reușită}
\label{sec:criterii-reusita}

\subsection{Calitatea analizei emoționale}
\begin{itemize}
  \item \textbf{Precizia identificării} - sistemul să atingă o \emph{performanță satisfăcătoare} în recunoașterea corectă a emoțiilor
  \item \textbf{Echilibrul în detectare} - capacitatea de \textit{identificare echitabilă} a tuturor tipurilor de emoții analizate
\end{itemize}

\subsection{Experiența utilizatorului}
\begin{itemize}
  \item \textbf{Răspuns rapid} - analiza unei opere să se finalizeze în \emph{maximum 15 secunde}
  \item \textbf{Interacțiune fluidă} - interfața să devină \textit{funcțională} în maximum 10 secunde de la pornire
\end{itemize}

\subsection{Completitudinea livrabilelor}
\begin{itemize}
  \item \textbf{Vizualizări comprehensive} - toate analizele să genereze \emph{reprezentări grafice} și \textit{tabele explicative}
  \item \textbf{Flexibilitate în prezentare} - disponibilitatea \emph{opțiunilor de export} și a \textit{narațiunilor descriptive} pentru fiecare analiză
\end{itemize}

\section{Figuri explicative}
\label{sec:figuri-explicative}

Prin \cref{fig:obj-schematic,fig:val-flow} sunt indicate locuri pentru două figuri de sinteză; acestea vor fi completate în capitolele tehnice, cu imagini exportate din pipeline-ul de validare și schema de sistem.

\begin{figure}[htbp]
  \centering
  \includegraphics[width=0.9\linewidth]{Obiectivele.jpg}
  \caption{Obiective și funcționalități principale: intrare → profil emoțional multi-etichetă (cu praguri adaptive) → ieșiri și prezentare.}
  \label{fig:obj-schematic}
\end{figure}

% Placeholder pentru fluxul de validare (va putea fi înlocuit cu figura generată din pipeline în Cap. 6)
\begin{figure}[htbp]
  \centering
  \fbox{\parbox{0.85\linewidth}{\centering Placeholder flux validare / evaluare\\(va fi înlocuit cu exportul final din pipeline)\\Etape: Input imagini → Preprocesare → Model ansamblu → Scoruri emoții → Praguri adaptive → Metrici (F1, AP) → Raport}}
  \caption{Flux conceptual de validare și evaluare: traseul datelor de la imagine brută la metrici sintetice.}
  \label{fig:val-flow}
\end{figure}

\section{Sinteză}
\label{sec:obiective-sinteza}

\textbf{Obiectivele proiectului} sunt \emph{clare și măsurabile}, vizând dezvoltarea unei platforme complete pentru analiza emoțională a operelor de artă. Prin combinarea \textit{preciziei tehnice} cu \emph{accesibilitatea pentru utilizatori}, platforma ArtAdvisor își propune să democratizeze înțelegerea și aprecierea artei.

Capitolul de față trasează \textbf{cadrul conceptual} și criteriile de realizare. Fundamentele teoretice, proiectarea și implementarea sistemului, precum și testarea și validarea vor fi detaliate în capitolele următoare, demonstrând atingerea obiectivelor propuse și \textit{utilitatea practică} a soluției dezvoltate.

